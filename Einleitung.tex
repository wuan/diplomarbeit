\chapter*{Einleitung und Motivation}
\addcontentsline{toc}{chapter}{Einleitung und Motivation}

AuIn$_2$ zeigt bei Temperaturen unterhalb 1~mK einige faszinierende kernmagnetische
Eigenschaften, die z.\ T. von der hohen Kernspinpolarisation der In"=Kerne initiiert 
werden. Die in \cite{Wagner_Dis} an verschieden vorbehandelten AuIn$_2$"=Proben durchgeführten
NMR"=Messungen zeigten ein schwer interpretierbares Verhalten, u.\ a.\ wurden stark
unterschiedliche Spin"=Spin Relaxationszeiten gemessen und an einer Probe sogar die Möglichkeit der Erzeugung
multipler Spin"=Echos festgestellt.

Mit den im Rahmen dieser Arbeit durchgeführten NMR"=Experimenten soll das zu AuIn$_2$ isostrukturelle
AuGa$_2$ untersucht werden. Die Vertreter der Klasse der intermetallischen Verbindungen AuX$_2$
(X=Al,Ga,In) liegen alle in der kubischen CaF$_2$ Struktur vor, in der die Atome des Elements X
auf einfach kubischen Gitterplätzen sitzen (siehe Abb.~\ref{fig:auga2}), was u.\ a.\ die
Unterdrückung der Quad\-ru\-pol\-auf\-spal\-tung der NMR"=Linien zur Folge hat.

Es ist von Interesse, inwieweit auch in \aug{} das Relaxationsverhalten der Ga"=Spins ähnlich zu
AuIn$_2$ durch die Spinpolarisation bei tiefen Temperaturen modifiziert wird und welche 
Erklärungen für das bei AuIn$_2$ beobachtete Verhalten gefunden werden können.

Wie schon seit den sechziger Jahren bekannt ist zeigt \aug, im Gegensatz zu den beiden anderen
AuX$_2$"=Verbindungen, aufgrund von Besonderheiten der Bandstruktur eine ungewöhnliche
Temperaturabhängigkeit der Suszeptibilität und der Knightshift, wobei letztere im
Temperaturbereich von 4.2 K bis 300 K sogar ihr Vorzeichen wechselt. \cite{AuGa2Dilemma}.

In \cite{Stephan_Dis} wird für \aug{} als Supraleiter 1. Art eine Sprungtemperatur von 1.05~K und
ein kritisches Feld von 7.6~mT bestimmt. Messungen der spezifischen Wärme bzw. der
kernmagetischen Resonanz an reinem Gallium haben gezeigt, daß für $T>500\;\mu$K keine
kernmagnetischen Anomalien auftreten. In \cite{Stephan_Dis} wurde die Ordnungstemperatur der
magnetischen Kermomente aufgrund der Dipol"=Dipol Wechselwirkung zwischen den Ga"=Kernen auf
$T_c^{DD}=0.20\;\mu$K abgeschätzt.

{\bfseries Hinweis: }
Im Weiteren wird $\vec B=\mu_r\mu_0\vec H$ das Magnetfeld genannt und in der Einheit "`Tesla"' angegeben.

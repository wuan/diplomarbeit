%
% Datei: Studium/Diplomarbeit.tex
%
% Begonnen am: 18/01/97
%
%\documentclass[twoside,12pt,a4paper]{report}
%\usepackage[latin1]{inputenc}
%\usepackage[english,german]{babel}
%\usepackage[T1]{fontenc}

% Computer Concrete Fonts verwenden
%\usepackage{ccfonts}
%\usepackage{times,mathptm}

%\usepackage{typearea}
%\typearea[2cm]{14}

%\makeatletter
%\renewcommand\normalsize{%
%   \@setfontsize\normalsize\@xivpt{18}%
%   \abovedisplayskip 14\p@ \@plus3\p@ \@minus7\p@
%   \abovedisplayshortskip \z@ \@plus3\p@
%   \belowdisplayshortskip 6.5\p@ \@plus3.5\p@ \@minus3\p@
%   \belowdisplayskip \abovedisplayskip
%   \let\@listi\@listI}
%\renewcommand\small{%
%   \@setfontsize\normalsize\@xiipt{14.5}%
%   \abovedisplayskip 12\p@ \@plus3\p@ \@minus7\p@
%   \abovedisplayshortskip \z@ \@plus3\p@
%   \belowdisplayshortskip 6.5\p@ \@plus3.5\p@ \@minus3\p@
%   \belowdisplayskip \abovedisplayskip
%   \let\@listi\@listI}
%
%\normalsize
%\makeatother

% Häufig benötigte Worte/Floskeln
%\newcommand{\aug}{\ifmmode{\mathrm{AuGa}_2}\else{AuGa$_2$}\fi}

%%%%%%%%%%%%%%%%%%%%%%%%%%%%%%%%%%%%%%%%%%%%%%%%%%%%%%%%%%%%%%%%%%%%%%%%%%%%%%%%%%%%%%%%%%%%%%%%%%

%\sloppy

\thispagestyle{empty}

%\begin{document}

%\setlength{\parindent}{0em}
%\setlength{\parskip}{1ex}

%\phantom{asdf}
%\newpage
%\phantom{asdf}
\begin{center}
 \vspace{1cm}
\large
{\Huge\bfseries Magnetische Kernspinresonanz\\[1cm]
 an \aug\\[1cm]
 bei tiefen Temperaturen\\[3cm]}

{\Large\bf  Diplomarbeit}\vspace{2cm}\\

vorgelegt\\
von\\
Andreas Würl\vspace{1cm}\\
Universität Bayreuth\\
Lehrstuhl Experimentalphysik V\vspace{1cm}\\
eingereicht am 2. März 1998 \vspace{1cm}\\
1. Gutachter: Prof. Dr. G. Eska \\
2. Gutachter: Prof. Dr. E. Rössler \vspace{1cm}\\
\vfill
Bayreuth, 1998
\end{center}
%%%%%%%%%%%%%%%%%%%%%%%%%%%%%%%%%%%%%%%%%%%%%%%%%%%%%%%%%%%%%%%%%%%%%%%%%%%%%%%%%%%%%%%%%%%%%%%%%%
%\end{document}

\newpage
\phantom{space}
\thispagestyle{empty}
